\documentclass{article}
\usepackage {color}
\usepackage {amsmath}
\usepackage {alltt}
\usepackage{graphicx} 
\usepackage{caption} 
\usepackage{fancyvrb,relsize}
\usepackage[latin1]{inputenc}

\begin{document}

\title{Computing With Tiles}
\author{
Rahul Gopinath\\
gopinath@eecs.oregonstate.edu
\and
Yonglei Zheng\\
zheng@eecs.oregonstate.edu
\and
Madhura Vadvalkar\\
vadvalkar@eecs.oregonstate.edu
\and
Junyuan Lin\\
lin@eecs.oregonstate.edu
\and
Liping Lin\\
lin@eecs.oregonstate.edu
}

\maketitle

% Hao Wang has shown that certain problems about tiling are RE complete. Can you use tilings to compute various functions, e.g how can you compute the Fibonacci numbers using tilings?

\begin{abstract}
Wang tiles were first proposed by the mathematician Hao Wang in 1961 \cite{wang}. These are four coloured tiles such that the four edges in a tile have different colours. These also have the property that two tiles can be laid edge to edge if and only if their edge colours match. The basic question about wang tiles are whether a given set of tiles can tile an infinte plane. In this paper we examine Wang tiles that are aperiodic\cite{undecide}. We also examine a way to encode functions into Wang tiles and as an example consider how we can encode a fibonacci number series into a Wang tiling \cite{tilings}.
\end{abstract}

\section*{Introduction}
Wang tiles are four coloured tiles that can be used to tile a plane.  They were first proposed by the mathematician Hao Wang in 1961. These are four coloured tiles such that the four edges in a tile have different colours. These also have the property that two tiles can be laid edge to edge if and only if their edge colours match. The number of tiles are fixed and are called prototyles. The basic question is whether the given number of prototiles can completely tile a plane. Tiling refers to arranging the tiles side by side on an infinite plane such that all adjuscent edges have the same colours, and there are no empty spots on the plain. This means that we allow the translation of the original prototiles but not rotation or mirroring.

Wang presented an algorithm that tried to decide if any finite set of prototiles could tile a plane. However this algorithm assumed that any set of tiles that were able to tile a plane would be periodic. However it was later shown that aperiodic tilings exist. We include in this paper the fibonacci tiling of plane using Wang tiles. Later Wang was able to show that Wang tiles could be used to simulate a Turing machine by considering each row being tiled as a tape in a single tape Turing machine such that the previous rows captured a history of the turing machine.

In this paper we also include one encoding of a turing machine that computes fibonacci numbers in Wang tiles.

\section*{Aperiodic Tiling}
It is possible to encode functions as aperiodic tilings using the Wang tilings. Below we present one of the possible ways to encode fibonacci series. (Explanation of our tilings)

\section*{Turing Machines}
It was shown later that it is possible to encode Turing machines into wang tiles.

\section*{Related Work}
Expand on the first T.M with 2000 tiles, reduced to 104, and later 13. Penrose tiling, Wolfram cellular automata. DNA computing, 3D wang tiles.

\section*{Simulation}
Our haskell simulation.

\section*{Conclusion}
In this paper we explored wang tiles that are able to tile a plane aperiodically. For this we used two different techniques. The first was direct tiling as given in \cite{tilings}, and later by simulating a Turing machine on the Wang tiles as given in \cite{tmtiles}.

\bibliographystyle{apalike}   % (uses file "plain.bst")
\bibliography{wang}       % expects file "myrefs.bib"
\end{document}

